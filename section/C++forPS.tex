\section{PS를 위한 C++}

\subsection{왜 C++?}

\begin{frame}{\textbf{\currentname}}
    \begin{block}{\texttt{C++}을 왜 쓸까요?}
        \begin{itemize}
            \item 빠릅니다.
            \item standard library가 강력합니다.
        \end{itemize}       
    \end{block}
    
    \begin{block}{\texttt{using namespace std;}}
        \begin{itemize}
            \item standard library에 있는 모든 함수와 클래스는 std안에 있습니다.
            \item 따라서 \texttt{using namespace std;}를 통해 써야할 \texttt{std::}를 생략할 수 있습니다.
        \end{itemize}
    \end{block}

    \sout{\footnotesize Python은 억까가 좀 있습니다.}
\end{frame}

\subsection{Input / Output}
\begin{frame}{\textbf{\currentname}}
    \begin{block}{\texttt{std::cin, std::cout}}
        \begin{itemize}
            \item 입출력 시 formating이 필요 없습니다.
            \item \texttt{cin >> a;} \texttt{cout << a;}
        \end{itemize}
    
    \end{block}
\end{frame}

\subsection{bits/stdc++.h}

\begin{frame}{\textbf{\currentname}}
    \begin{block}{\texttt{bits/stdc++.h}}
        \begin{itemize}
            \item GCC 컴파일러 상에 포함되어 있는 모든 헤더 파일을 포함하는 헤더 파일입니다.
            \item \texttt{\#include <bits/stdc++.h>} 
            \item 초보자인 여러분께는 권장하지 않습니다.
        \end{itemize}
    \end{block}

    \begin{block}{각 헤더엔 뭐가 있나요?}
        \begin{itemize}
            \item 각 헤더(\texttt{algorithm, queue, string...}마다 모두 정의되어있는 함수와 자료구조들이 있습니다. 차차 알아갑시다.
            \item \texttt{sort(a,a+n); lower\_bound(a.begin(), a.end(), x)} 등 같이 알아가봅시다.
        \end{itemize}
    \end{block}
\end{frame}

\subsection{standard library와 vector}
\begin{frame}{\textbf{\currentname}}
    \begin{block}{\texttt{std::vector}}
        \begin{itemize}
            \item 자동으로 메모리가 할당되는 배열
            \item \texttt{vector<int> a;}로 선언합니다.
            \item \texttt{a.push\_back(3); a.pop\_back(); a.size();} 등
            \item 아주 아주 아주 아주 많이 쓰입니다.
            \item \href{https://blockdmask.tistory.com/70}{참고}
        \end{itemize}
    \end{block}
\end{frame}

\subsection{FastIO}
\begin{frame}[fragile]{\textbf{\currentname}}
    \vspace{0.5cm}
    
    \begin{block}{Fast Input/Output의 필요성}
        \begin{itemize}
            \item 개행을 의미하는 \texttt{std::endl}의 경우, buffer flush가 일어납니다.
            \item 실행 시간에 많은 부분을 차지하므로, 이 기능을 끕니다.
            \item 위 문장들을 입력한 뒤에는 \texttt{scanf/cin}과 \texttt{printf/cout}을 혼용하면 안됩니다.
        \end{itemize}
    \end{block}

    \begin{block}{사용 방법}
        \begin{lstlisting}[language=C++, xleftmargin=10pt]
std::ios_base::sync_with_stdio(false);
std::cin.tie(nullptr);
std::cout.tie(nullptr);
\end{lstlisting}
    \end{block}
\end{frame}