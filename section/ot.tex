\section{OT}

\subsection{강의 목표}
\begin{frame}{\textbf{\currentname}}
    스터디가 끝나면 여러분들이

    \begin{itemize}
        \item 전필 과목(자료구조, 알고리즘)을 \underline{아주 쉽게}
        \item 기업 코딩 테스트 \underline{어렵지 않게}
    \end{itemize}

    해결할 수 있는 수준을 가질 수 있도록 내용 구성을 하였습니다.
\end{frame}

\subsection{강의 목차}
\begin{frame}{\textbf{\currentname}}

\vspace{0.4cm}

\renewcommand{\arraystretch}{1.4}
\begin{table}[]
\resizebox{\textwidth}{!}{%
\begin{tabular}{|c|c|cc|}
\hline
\rowcolor[HTML]{DC5958} 
{\color[HTML]{FFFFFF} 주차} & {\color[HTML]{FFFFFF} 주제} & \multicolumn{1}{c|}{\cellcolor[HTML]{DC5958}{\color[HTML]{FFFFFF} 1회차 내용}} & {\color[HTML]{FFFFFF} 2회차 내용} \\ \hline
1                         & PS 기초                     & \multicolumn{1}{c|}{OT, 빠른 입출력, PS를 위한 C++}                                & 시간복잡도, 정렬, 이분탐색               \\ \hline
2                         & 기초 수학                     & \multicolumn{1}{c|}{소수 판별, 소인수 분해,최대 공약수}                                  & 기초 정수론 및 조합론                  \\ \hline
3                         & 완전탐색 및 재귀                 & \multicolumn{1}{c|}{브루트 포스, 재귀, 백트래킹}                                      & 재귀, 백트래킹, 분할 정복               \\ \hline
4                         & 자료구조 기초                   & \multicolumn{1}{c|}{스택, 큐, 덱}                                              & 집합과 맵, 해싱, 좌표 압축, 우선순위 큐      \\ \hline
5                         & 그래프 이론                    & \multicolumn{1}{c|}{그래프의 표현, DFS/BFS 기초}                                   & DFS/BFS 심화                    \\ \hline
6                         & 다이나믹 프로그래밍                & \multicolumn{1}{c|}{DP 개념 : 오토마타, 상태전이, 최적부분구조}                            & DP 응용 : LIS, LCS, Knapsack    \\ \hline
7                         & 그리디, 애드 혹                 & \multicolumn{2}{c|}{\cellcolor[HTML]{FFFFFF}그리디 증명과 응용, 애드 혹}                                              \\ \hline
8                         & 자료구조 심화 및 기타 테크닉          & \multicolumn{1}{c|}{BST, 우선순위 큐(복습), 분리집합}                                 & 투 포인터, 비트마스킹                  \\ \hline
9                         & 최단거리 알고리즘                 & \multicolumn{2}{c|}{다익스트라, 밸만포드, 플로이드워셜}                                                                   \\ \hline
10                        & 최종 정리 및 콘테스트              & \multicolumn{2}{c|}{자체 내부 콘테스트}                                                                            \\ \hline
\end{tabular}%
}
\end{table}
\end{frame}
